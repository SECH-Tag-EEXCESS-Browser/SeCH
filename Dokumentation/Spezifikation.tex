%\part{Spezifikation}

\chapter{Browser}
Im Folgenden werden die Use--Case-Diagramme des \SECH-Browsers übersichtlich vorgestellt. Diese veranschaulichen alle verschiedenen Funktionalitäten, die der Benutzer tätigen kann.

\section{Menüführung}
\begin{figure}[htb]
\includegraphics[width=\textwidth]{Use-Case-Diagramme_001.pdf}
	\caption{Navigationselemente}
	\label{fig:Navigationselemente}
\end{figure}
Dieses Use--Case-Diagramm zeigt die wesentlichen Funktionen des Browsers an. Der Nutzer kann auf einer Homepage eine Seite zurück- und vorspringen. Weiterhin hat er die Möglichkeit die Lesezeichentabelle auszufahren und im Anschluss diese wieder einzufahren.

\begin{figure}[htb]
\includegraphics[width=\textwidth]{Use-Case-Diagramme_003.pdf}
	\caption{Navigationselemente}
	\label{fig:Navigationselemente}
\end{figure}
In diesem Use--Case-Diagramm werden weitere Navigationselemente des Browsers vorgestellt. Der Nutzer kann eigene Lesezeichen hinzufügen, die von ihm definierte Startseite laden, eine URL in die Addressbar eingeben und diese dann laden, die aktuelle Seite neu laden, die \SEARCH-Tabelle aus- und einfahren und das Fenster für Optionen öffnen.

\section{Lesezeichenverwaltung}
\begin{figure}[htb]
\includegraphics[width=\textwidth]{Use-Case-Diagramme_002.pdf}
	\caption{Lesezeichenverwaltung}
	\label{fig:Lesezeichenverwaltung}
\end{figure}
Ist die Lesezeichentabelle ausgefahren, so kann der Benutzer ein persönlich angelegtes Lesezeichen auswählen und es wird die von ihm gewünschte Seite geladen. Das Löschen und das Editieren ausgewählter Lesezeichen ist ebenfalls in der Tabelle verwirklichbar.

\section{\SEARCH-Tagverwaltung}
\begin{figure}[htb]
\includegraphics[width=\textwidth]{Use-Case-Diagramme_004.pdf}
	\caption{\SEARCH-Tagverwaltung}
	\label{fig:SEARCH-Tagverwaltung}
\end{figure}
Der Nutzer kann durch zwei verschiedene Wege Suchergebnisse für das gewünschte \SEARCH-Tag anzeigen lassen. Zunächst muss der Nutzer durch Eingabe und Bestätigung einer URL die Seite aufrufen. Die erste Variante ist das gewünschte \SEARCH-Tag, welches hervorgehoben wird, anzuklicken. Die zweite Variante ist durch Ausfahren der \SEARCH-Tabelle das gewünschte \SEARCH-Tag auszuwählen. Beide Wege führen zum selben Ergebnis und zwar die Anzeige des ersten Suchergebnisses des jeweiligen \SEARCH-Tags. Zusätzlich kann der Nutzer alle weitere Suchergebnisse zu einem \SEARCH-Tag anzeigen lassen.

\section{Einstellungen}
\begin{figure}[htb]
\includegraphics[width=\textwidth]{Use-Case-Diagramme_005.pdf}
	\caption{Einstellungen}
	\label{fig:Einstellungen}
\end{figure}
Befindet sich der Nutzer in den Browsereinstellungen, so hat er die Möglichkeit bestimmte Suchmaschinen, welche für die Suchergebnisse der \SEARCH-Tags verwendet werden, zu aktivieren oder diese zu deaktivieren. Im Weiteren kann der Benutzer sein persönliches Nutzerprofil verwalten und seine Startseite für die Applikation festlegen, welche direkt nach dem Öffnen der Anwendung als erste Seite geladen wird.
