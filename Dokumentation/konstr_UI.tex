%\part{Konstruktion}

\chapter{User Interface}

Im Folgenden wird ein Überblick über die Klassenabhängigkeiten des UIs gegeben, der Verlauf zur Anzeige eines \SEARCH-Links erklärt, sowie die Packagebedeutungen innerhalb des UIs erläutert.

\begin{figure}[h]
	\centering
	\includegraphics[width=\textwidth]{UI_Uebersicht_Klassenabhaengigkeiten}
	\caption{Klassenabhängigkeiten im UI}
	\label{fig:Klassenabhängigkeiten}
\end{figure}

Die obige Darstellung zeigt die Verbindungen der Klassen des User Interfaces. Hierbei steht der ViewController im Mittelpunkt, da von ihm die meiste Funktionalität ausgeht.

\begin{figure}[ht]
	\centering
	\includegraphics[width=\textwidth]{SearchTagAnzeige}
	\caption{Searchtag Anzeige}
	\label{fig:Searchtag Anzeige}
\end{figure}

Obige Grafik zeigt den Programmablauf beim Klick auf einen Searchtag. Die Position des Klicks wird über JavaScript ausgelesen und an den WkWebview weitergeleitet. Dieser erzeugt ausgehend von der übertragenen Position einen PopView und zeigt diesen an. 

%\part{Konstruktion}
%\chapter{User Interface}

\section{ViewController}

\subsection{SearchTableViewController}
\subsection{ViewController}
\subsection{PopViewController}
\subsection{SettingsController}
\subsection{main.js}
\subsection{readHead.js}

%\subsubsection{Unterteilabschnitt}
%\paragraph{Paragraph}
%\subparagraph{Unterparagraph}

\newpage
%\part{Konstruktion}
%\chapter{User Interface}

\section{Delegate}

Die Delegates im \SECH-Browser dienen zur Bereitstellung einer einheitlichen Struktur innerhalb des Programms. Von den einzelnen Delegate Klassen werden Funktionen zur Darstellung von Informationen oder interne Protokolle aufgerufen. Im \lstinline|WebViewDelegate| werden die Funktionen zur Validierung der Website mit Hilfe der JavaScript Dateien und die Extraktion der \SEARCH-Tags angestoßen.

Die \lstinline|readHead.js| enthält den Javascript Code zur Auslesen des Headbereiches einer HTML-Seite. In der Klasse \lstinline|FavTableDelegate| werden die Lesezeichen bereitgestellt. Der Benutzer hat die Möglichkeit ein ausgewähltes Lesezeichen auszuwählen und es sich im Browser anzeigen zulassen. Zusätzlich enhält die Klasse die Funktionen, um Lesezeichen bearbeiten und löschen zu können. Der SechTableDelegate enthält die Routine zum Darstellen von \SEARCH-Links aus der \SECH-Tabelle. 
\subsection{AppDelegate}
\subsection{WebViewDelegate}
\subsection{readHead.js}
\subsection{FavTableDelegate}
\subsection{SechTableDelegate}

%\subsubsection{Unterteilabschnitt}
%\paragraph{Paragraph}
%\subparagraph{Unterparagraph}

\newpage
%\part{Konstruktion}
%\chapter{User Interface}

\section{DataSource}
Die DataSources im \SECH-Browser dienen zur Bereitstellung der Daten in den Tabellen. Von den einzelnen Klassen werden jeweils Funktionen zum befüllen der einzelnen Zellen aufgerufen. In der Klasse FavTableDataSource werden alle gespeicherten Lesezeichen geladen und danach in die einzelnen Zellen der Tabelle geladen. Im SechTableDataSource werden alle \SEARCH-Tags die auf der Seite gefunden wurden in den Zellen der SechTabelle dargestellt. Zusätzliche befinden sich noch Funktionen zum Erweitern der \SEARCH-Tags in der Klasse.

\subsection{FavTableDataSource}
\subsection{SechTableDataSource}

%\subsubsection{Unterteilabschnitt}
%\paragraph{Paragraph}
%\subparagraph{Unterparagraph}


\newpage
%\part{Konstruktion}
%\chapter{User Interface}

\section{Components}

\subsection{FavoriteCell}
\subsection{AdressBar}
\subsection{SearchCell}

%\subsubsection{Unterteilabschnitt}
%\paragraph{Paragraph}
%\subparagraph{Unterparagraph}

\newpage
%\part{Konstruktion}
%\chapter{User Interface}
%\section{Packages}
\subsection{Persistence}

Im Persistence-Package befinden sich Models die für eine persistente Speicherung gedacht sind. Das \lstinline|FavouritesModel| stellt einerseits das nötige Model, andererseits die Funktion des Speicherns für Lesezeichen zur Verfügung.

\subsubsection{FavoritesModel}

%\paragraph{Paragraph}
%\subparagraph{Unterparagraph}


\newpage
%\part{Konstruktion}
%\chapter{User Interface}

\section{WebContent}

Der WebContent stellt die Schnittstelle zwischen dem UI und der SearchExtraction dar. Er beinhaltet HTML-Code, in Form eines Strings, sowie die URL, von welcher der HTML-Code stammt. Der HTML-Code ist in den Head und den Body der Website geteilt.\newline
Aus dem WebContet wird in der SearchExtraction ein SearchModel erzeugt.

%\subsection{Teilabschnitt}
%\subsubsection{Unterteilabschnitt}
%\paragraph{Paragraph}
%\subparagraph{Unterparagraph}


