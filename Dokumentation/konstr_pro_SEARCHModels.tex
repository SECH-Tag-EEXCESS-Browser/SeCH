%\part{Konstruktion}
%\chapter{Programmlogik}

\section{SEARCHModels}
\lstinline|SEARCHModels| stellt die Schnittstelle zwischen der \lstinline|SEARCHExtraction| einerseits und dem \lstinline|TaskController| andererseits dar. Von der \lstinline|SEARCHExtraction| erzeugt, beinhaltet ein \lstinline|SEARCHModels|-Objekt eine Sammlung von \lstinline|SEARCHModel|-Objekten, erzeugt aus dem Inhalt einer Website. In einem \lstinline|SEARCHModel|-Objekt werden \SEARCH-Tag-Links mit der zugehörigen \SEARCH-Tag-Section und/oder dem \SEARCH-Teg-Head sowie möglichen Filtern gespeichert.

Zur globalen Identifikation wird in einem \lstinline|SEARCHModel| die URL der Website von der es kommt gesetzt. Um ein \lstinline|SEARCHModel| innerhalb einer Seite zu finden wird ein Index bezüglich der Position in der Seite gesetzt.