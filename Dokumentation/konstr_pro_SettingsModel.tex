%\part{Konstruktion}
%\chapter{Programmlogik}

\section{SettingsModel}
Im \lstinline|SettingsModel| werden die verschiedenen Einstellungen für den Browser gespeichert. Der Nutzer kann diese nicht in der App, sondern über das Einstellungsmenü des Gerätes ändern.

\subsection{Einstellungen}
Die Einstellungen sind in zwei verschiedene Kategorien eingeteilt.
\begin{itemize}
     \item Nutzerprofil
     \item Browsereinstellungen
\end{itemize}

\subsubsection{Nutzerprofil}  
Hier können die Nutzerdaten angegeben werden. Davon wird bisher nur die Sprache verwendet.

Einstellungsmöglichkeiten:
\begin{itemize}  
     \item Name  
     \item Alter  
     \item Stadt
     \item Land
     \item Sprache
\end{itemize}

\subsubsection{Browsereinstellungen}  
Hier kann die Startseite des Browser festgelegt werden. Die hier gesetzte Seite wird noch nicht im Browser als Startseite übernommen.

\subsubsection{Suchmaschinen}  
Geplant ist eine weitere Kategorie, \lstinline|Suchmaschinen|. Hier kann der Nutzer dann die verschiedenen Suchmaschinen einstellen, die für die \SEARCH-Tags verwendet werden sollen. Dabei kann entweder jede Suchmaschine einzeln an- und/oder abgewählt werden, oder den Empfehlungen des Autors gefolgt werden.

Zur Auswahl sollen dann stehen:
\begin{itemize}  
     \item DuckDuckGo
     \item Eexcess
     \item Faroo  
\end{itemize}
Sobald eine Suchmaschine gewählt wurde, werden die Empfehlungen des Autors ignoriert.
