\section{SettingsModel}
Im SettingsModel werden die verschiedenen Einstellungen für den Browser gespeichert.
Der Nutzer kann diese über das Einstellungsmenü des Geräts ändern.

\paragraph{Einstellungen}  

Die Einstellungen sind in drei verschiedene Kategorien eingeteilt.
\begin{itemize}  
     \item Suchmaschinen  
     \item Nutzerprofil
     \item Browsereinstellungen
\end{itemize}

\subparagraph{Suchmaschinen}  

Hier kann der Nutzer die verschiedenen Suchmaschinen einstellen, die für die SearchTags verwendet werden sollen.
Dabei kann entweder jede Suchmaschine einzeln gewählt werden, oder den Empfehlungen des Autors gefolgt werden.
Zur Auswahl stehen:
\begin{itemize}  
     \item DuckDuckGo
     \item Eexcess
     \item Faroo  
\end{itemize}
Sobald eine Suchmaschine gewählt wurde, werden die Empfehlungen des Autors ignoriert.

\subparagraph{Nutzerprofil}  
Hier können die Nutzerdaten angegeben werden.
Davon wird bisher nur die Sprach verwendet.
Einstellungsmöglichkeiten:
\begin{itemize}  
     \item Name  
     \item Alter  
     \item Stadt
     \item Land
     \item Sprache
\end{itemize}

\subparagraph{Startseite}  
Hier kann die Startseite des Browser festgelegt werden.
Die hier gesetzte Seite wird nocht nicht im Browser als Startseite übernommen.
