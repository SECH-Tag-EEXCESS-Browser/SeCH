%\part{Konstruktion}
%\chapter{User Interface}
%\section{Packages}

\subsection{DataSource}
Die DataSources im \SECH-Browser dienen der Bereitstellung der Daten in den Tabellen. Von den einzelnen Klassen werden jeweils Funktionen zum Befüllen der einzelnen Zellen aufgerufen.

In der Klasse \lstinline|FavTableDataSource| werden zunächst alle gespeicherten Lesezeichen geladen. Danach werden die einzelnen Zellen der Tabelle mit den Lesezeichen befüllt.

Im \lstinline|SechTableDataSource| werden alle \SEARCH-Tags die auf der Seite gefunden wurden in den Zellen der \SECH-Tabelle dargestellt. Weiterhin befinden sich noch Funktionen zum Erweitern der \SEARCH-Tags in der Klasse.