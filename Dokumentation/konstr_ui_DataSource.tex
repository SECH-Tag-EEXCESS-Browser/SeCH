%\part{Konstruktion}
%\chapter{User Interface}

\section{DataSource}
Die DataSources im \SECH-Browser dienen zur Bereitstellung der Daten in den Tabellen. Von den einzelnen Klassen werden jeweils Funktionen zum befüllen der einzelnen Zellen aufgerufen. In der Klasse FavTableDataSource werden alle gespeicherten Lesezeichen geladen und danach in die einzelnen Zellen der Tabelle geladen. Im SechTableDataSource werden alle \SEARCH-Tags die auf der Seite gefunden wurden in den Zellen der SechTabelle dargestellt. Zusätzliche befinden sich noch Funktionen zum Erweitern der \SEARCH-Tags in der Klasse.

\subsection{FavTableDataSource}
\subsection{SechTableDataSource}

%\subsubsection{Unterteilabschnitt}
%\paragraph{Paragraph}
%\subparagraph{Unterparagraph}

