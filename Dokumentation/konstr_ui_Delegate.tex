%\part{Konstruktion}
%\chapter{User Interface}

\section{Delegate}

Die Delegates im \SECH-Browser dienen zur Bereitstellung einer einhaltlichen Struktur innerhalb des Programms. Von den einzelnen Delegate Klassen werden Funktionen zur Darstellung von Informationen oder interne Protokolle aufgerufen. Im WebViewDelegate werden die Funktionen zur Validierung der Website mit Hilfe der JavaScript Dateien und die SearchExtraction angestoßen.

Die readHead.js enthält den Javascript Code zur Auslesen des Headbereiches einer HTML-Seite. In der Klasse FavTableDelegate werden die Lesezeichen bereitgestellt. Der Benutzer hat die Möglichkeit ein ausgewähltes Lesezeichen auszuwählen und es sich im Browser anzeigen zulassen. Zusätzlich enhält die Klasse die Funktionen, um Lesezeichen bearbeiten und löschen zu können. Der SechTableDelegate enthält die Routine zum Darstellen von \SEARCH-Links aus der \SECH-Tabelle. 
\subsection{AppDelegate}
\subsection{WebViewDelegate}
\subsection{readHead.js}
\subsection{FavTableDelegate}
\subsection{SechTableDelegate}

%\subsubsection{Unterteilabschnitt}
%\paragraph{Paragraph}
%\subparagraph{Unterparagraph}
