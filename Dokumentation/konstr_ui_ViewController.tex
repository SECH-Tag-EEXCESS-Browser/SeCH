%\part{Konstruktion}
%\chapter{User Interface}

\section{ViewController}
Der \lstinline|ViewController| dient zur Steuerung der Funktionen welche direkt mit dem Interface zusammen gehören. Dazu gehören
\lstinline|IBActions| welche von Buttons oder anderen Interkationsflächen angestoßen werden können, sowie Funktionen welche direkt mit diesen zusammenhängen, beziehungsweise aus diesen resultieren. Der \lstinline|ViewController| ist die zweite Schicht nach dem \lstinline|UI|. Im
Folgenden werden die einzelnen \lstinline|ViewController| mit ihren jeweiligen Funktionen beschrieben.
Der \lstinline|Viewcontroller| ist der Hauptviewcontroller für die normale Browseransicht mit der Adresszeile und dem \lstinline|WkWebView|. Er verwaltet die Kopfzeile mit Adresszeile sowie den Buttons für Vorwärts, Zurück, Lesezeichen, Lesezeichen hinzufügen, Reload, \SECH-Tabelle sowie die Settings. Die datei \lstinline|Main.js| welche aus dem \lstinline|Viewcontroller| heraus aufgerufen wird, markiert alle gefundenen \SEARCH-Tags. Außerdem wird im Falle eines Klicks die Position sowie die id des \SEARCH-Tags übermittelt.
Der \lstinline|PopViewController| ist zuständig für die Verwaltung eines neuen \lstinline|PopViews|, beim klicken auf einen \SEARCH-Tag. Er verwaltet die Anzeige des ausgewählten \SEARCH-Links in einem neuen \lstinline|WkWebView| sowie den Button für die Auswahl der verschiedenen Suchergebnisse zu einem \SEARCH-Tag.
Der \SEARCH-TableViewController verwaltet die einzelnen Suchergebnisse zu einem \SEARCH-Tag. Alle Ergebnisse werden sortiert in einer Tabelle mit Bild, Title und Link angezeigt.
Der \lstinline|SettingsController| ist für die Verwaltung der Browsereinstellungen zuständig.


%\subsubsection{Unterteilabschnitt}
%\paragraph{Paragraph}
%\subparagraph{Unterparagraph}
